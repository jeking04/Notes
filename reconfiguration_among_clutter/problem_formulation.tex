\documentclass[11pt, oneside]{article}   	% use "amsart" instead of "article" for AMSLaTeX format
\usepackage{geometry}                		% See geometry.pdf to learn the layout options. There are lots.
\geometry{letterpaper}                   		% ... or a4paper or a5paper or ... 
%\geometry{landscape}                		% Activate for for rotated page geometry
%\usepackage[parfill]{parskip}    		% Activate to begin paragraphs with an empty line rather than an indent
\usepackage{graphicx}				% Use pdf, png, jpg, or eps� with pdflatex; use eps in DVI mode
								% TeX will automatically convert eps --> pdf in pdflatex		
\usepackage{amssymb}
\usepackage{amsmath}

\title{Notes on Reconfiguration Among Clutter}
%\date{}							% Activate to display a given date or no date

\begin{document}
\maketitle
\section*{Introduction}
We wish to build on Mehmet's previous work on reconfiguration among clutter and design a planner which will search the space of possible pushing trajectories to search for a valid grasp.  This document serves  as a scratch pad for formulating the problem.

\section*{Assumptions}
The following list enumerates the assumptions/assertions we are making throughout the formulation of this problem:
\begin{enumerate}
\item We assume we are planning with a floating end-effector.
\item We have obstacles in our environment that don't move and should be avoided by the end-effector.
\item Hitting an object doesn't affect trajectory of an action.
\end{enumerate}

\section*{Naive Problem Formulation}
\subsection*{Representation of State}
Define the following:
\begin{enumerate}
\item $q_t$ - an $n$-dimensional vector describing the configuration of the end-effector at time $t$
\item  $O_t^i$ - the 3-dimensional ($x$, $y$, $\theta$) which represents the configuration of object $i$ at time $t$
\item $g_t$ - the 3-dimensional ($x$, $y$, $\theta$) configuration of the goal object at time $t$
\end{enumerate}
Then we can define the state at time $t$, $s_t$, by the vector:
\[ s_t = \left[\begin{array}{ccccc}
q_t & g_t & O_t^1 & ... & O_t^k
\end{array}
\right] \]

\subsection{Representing the Goal}\label{sec:graspability}
We can define the start state as the initial pose of the arm, initial pose of the goal object and the initial pose of all obstacles at a particular snapshot in time.  Defining the goal is less straightforward.  We can say that the goal is a function only of the pose of the goal object and the arm configuration.  Thus we can define a function:
\[ f(q_t, g_t) 
\begin{cases}
=0, & \textbf{the goal is graspable} \\
\neq 0, & \textbf{otherwise} 
\end{cases}
\]
In particular, let $S$ be the set of all poses of the end-effector in goal object frame that result in a valid grasp of the object. 
Then we want to check the following:
\[ q' = T^w_0F(q) \in S \]
where $F(q)$ computes the pose of the end-effector when the arm is in configuration $q$ via use of forward kinematics and $T_0^w$ defines the transform from world frame to goal object frame.
\newline\newline
We can sample valid grasps using the concept of TSRs.  A TSR is comprised of a matrix, $Bw$, which expresses valid bounds of values for the pose of the end-effector in object frame:
\[ Bw = \left[ \begin{array}{cc}
x_{min} & x_{max} \\
y_{min} & y_{max} \\
z_{min} & z_{max} \\
\alpha_{min} & \alpha_{max} \\
\beta_{min} & \beta_{max} \\
\theta_{min} & \theta_{max} \\
\end{array}\right] \]
We can now sample $x, y, z, \alpha, \beta, \theta$ values from the valid bounds represented by $Bw$.  
\newline\newline
Note: We will need to ensure that the preshape of the hand during pushing is such that we can close the hand without fear of catching other obstacles.  Otherwise, we will have to do a collision check here.

\subsection*{Action Space}
We want to search over our action space to find a series of actions that, when strung together, will provide a path for the end-effector to grasp the goal object.  We can define an action using the following parameters:
\begin{enumerate}
\item $\dot{v}$ -  The forward velocity of the end-effector.
\item $\dot{\theta}$ - The rotational velocity of the end-effector.
\item $\dot{a}$ - The hand aperture.
\item $\Delta t$ -The duration of the action.
\end{enumerate}
For each action, we precompute the volume of space swept by the end-effector during execution of the movement.  

\subsection*{Forward Planning with Graph Search}
For performing forward search we define the following mapping:
\[ s' = T(s, a) \]
Here $T$ forward simulates action $a$ starting from state $s$ and ends at state $s'$. We will use the physics simulator developed by Mehmet/Koval to simulate the action and any collisions between the end-effector and objects in the state space.  If an object is pushed into contact with either another object or an obstacle, the action will be consider invalid and the edge will not be added to the search graph.

We evaluate the function $f(q_{s'}, g_{s'})$ to determine when the search can be stopped.Details of this check were explained in section ~\ref{sec:graspability}.

We will turn the state space into a set of discrete values in each dimension and snap to the center of the nearest grid cell during execution.   This will allow us to recognize when we are exploring a region of state space that we have previously explored.  

\subsection*{Backward Planning with Graph Search}
Backward simulation of an action is less trivial than forward simulation.  To do this, we can plan the actions in reverse, as though they ended in our current cell.  For performing backward search, we must define the following mapping:
\[ s = T^{-1}(s', a) \]
We note that unlike forward search, this mapping will often be one to many.  In particular, we identify the following possible scenarios during back propagation of an action:
\begin{enumerate}
\item If, when the push is backward simulated, there is an object in the footprint, don't add the edge to the graph. This action was not a possible predecessor, as the obstacle in the footprint would have been removed during action execution.
\item If there are no object touching the footprint, we can just add the state that starts the action as a predecessor without changing the pose of any obstacles in the scene.  In this case we have a 1-to-1 mapping between state and predecessor.
\item If an object is on the edge of the footprint we add multiple states: the first represents the object not being moved, the others represent various locations where the object could have started and been pushed to the final location.  Here we must prove that we can enumerate all possible locations for the start of the object.
\end{enumerate}

\subsection*{Distance Functions}
If we want to use this formulation with a BiRRT planner we must define a valid distance function.  We examine the use of the following:
\[ d(T1_q^w, T2_q^w) \]
where $T_q^w$ defines the end-effector in goal object frame.
TODO: Try to prove this is valid and good.
\section*{Feasible Worlds}
We identify the following statements which much hold for a world be feasible:
\begin{enumerate}
\item No contact between any object and an obstacle
\item No contact between any object and another object
\item No contact between the end-effector and an obstacle
\end{enumerate}

\section*{Trivial Scenario}
We will first implement and test a fairly trivial scenario to prove the concept and get a feel for scenes where this is useful and possible gotchas.   The following describes the scenario:
\begin{enumerate}
\item Point robot - described by $x$, $y$ location
\item Point obstacles - again described by $x$, $y$ location
\item Goal - The goal is to get the point robot within 1 grid cell in each of the 4 cardinal directions of the goal item - which is a point described by $x$, $y$ location
\item Action set involves movements in only the cardinal directions.
\item If an obstacle is hit during action execution it moves in a direction +90 degrees from the direction of the action by 1 single grid cell.
\end{enumerate}

\section*{Open Questions}
The following is a list of questions to consider:
\begin{enumerate}
\item What coordinate frame should state be represented in? Seems like the end-effector frame could be useful.  Then all obstacles would be represented as distance from the end-effector which could simplify collision checking.
\item Pras mentioned the sorting problem.  Could we use this framework to sort via pushing? Is that interesting/novel.
\end{enumerate}

\section*{Related Reading}
\begin{enumerate}
\item "Two level RRT Planning for Robotic Push Manipulation" Zito, et. al  IROS 2012
\begin{itemize}
\item Present a two-level approach for push planning
\item The first level is an RRT which plans in the configuration space of the object being pushed
\item The second level is a depth first search which links pushes together to connect a sampled node to the search graph
\begin{enumerate}
\item Once a node is sampled for RRT, N candidate pushes are selected
\item The one which moves closest to the sampled node is executed until the distance to the sampled node decreases, then N new candidate pushes are selected and the process repeats until we are very close to the sampled node.
\item A push is checked by planning a straight line path in confiugraiton space that goes through the object, then forward simulating the execution of that path and the resulting push using a physics simulator.  After the action has been simulated for one time step, the distance to the sampled node is checked.  If it has decreased, the action along that straight line continues to be simulated. If it hasn't decreased, N new candidate pushes starting at the current position of the pushed object are generated and the process continues.
\end{enumerate}
\end{itemize}
\item "Achievable Push-Manipulation for Complex Passive Mobile Objects using Past Experience" Mericil, et. al. AAMAS 2013
\begin{itemize}
\item Uses learning and RRTs to build a planner for pushing in worlds where quasi-static assumption does not hold
\item First, a set of object trajectories is learned using the following steps
\begin{enumerate}
\item Pick m random push configurations, where a configuration is defined by a push point, a push direction and a duration
\item Collect n samples of execution of each of the m push configurations
\item Model the final pose of the object at the end of the push with a normal distribution
\item Remember these m models of object trajectories
\item Pick several goals in the task environment and try to plan to them in order to test how useful the set of learned trajectories is so far
\item Repeat
\end{enumerate}
\item During planning , use the memorized actions during the extend portion of the RRT
\item Uses an interesting distance function:
\[\sim(p_1, p_2) = \frac{d_{max}}{d(p_1, p_2)} cosSim(\hat{p_1}, \hat{p_2})\]
 where $d_{max}$ is the max distance in the task environment, $d(p_1, p_2)$ is the Euclidean distance between the two poses and $cosSim(\hat{p_1}, \hat{p_2})$ is the cosine similarity between the two poses assuming they are unit vectors from the origin.
\end{itemize}
\item "Disassembly Path Planning for Complex Articulated Objects" Cortes, Jaillet, Simeon 2007
\begin{itemize}
\item Present an RRT algorithm for disassembly of objects with articulated pars
\item Break the state space into 2 pieces: active and passive
\item Initially sample poses from the active piece of the state space and extend by making changes to active state
\item If passive state space impedes extension, continue extension by making changes to the passive portion of the state
\end{itemize}
\end{enumerate}

\section*{Random Ideas that involve related reading}
\subsection*{3 - Disassembly Path Planning for Complex Articulated Objects}
We could possible make a similar separation of our state space.  In particular we would say:
\[q_{act} = q_t \]
\[q_{pas} =  \left[O^1,...,O^k, g\right]\]
Then at time $t$ we could do the following to grow the backward tree:
\begin{enumerate}
\item Sample a configuration $q_{act}^{rand}$ which represents only a change to the end-effector.
\item Find the nearest configuration using a distance function based only on the end-effector - for example, euclidean distance between end-effector poses.
\item Backward simulate an action. Stop the action if a collision is detected, and return invalid.  Otherwise, label the configuration at the start of the action $q_{near}$ and add it as a node to the graph. 
\item During backwards simulation, generate a list of objects touching the footprint.
\item For each item $O^i$ in that list, generate a region $R^i$ where the object could have been pushed from. TODO: This is nontrivial, but perhaps we can use Mehmet's work in some way.
\item Sample a configuration $q_{pas}^{rand}$ which differs from $q_{near}$ only in the pose of object $O^i$. Call this new configuration $q_{near}$. Repeat until all items have been adjusted. Add $q_{near}$ as a node to the graph.
\end{enumerate}
\end{document}  


